\chapter{Requirements Analysis}
\label{cap:RequirementsAnalysis} 


\section{Actors}

As said before, the main actor is usually a single user which uses the ap-
plication on two different mobile phones, one being the stolen/lost one and
the other the rescue one. It’s also possible a second actor to exists: a (very)
trusted friend may be the one who sends remote commands to the target tele-
phone; since a password authentication scheme is employed, it’s not possible
to do any abuse in this case.

\section{Functional Requirements}

The application allows the user do to the following tasks:

\begin{itemize}
 \item Initialization
 \item Mobile Phone Association
 \item Password Change
 \item Unilateral Deassociation
 \item Remote Localization
 \item Mobile Phone Mark
 \item Remote Alarm Triggering
 \item Local Alarm Off
 \item Perimeter Selection
\end{itemize}

\section{Non Functional Requirements}

The remote control of a cellphone is a critical activity and has many security
and privacy requirements: the next paragraphs show them briefly: for in-depth
explainations see the design section (4.3). %metti il riferimento latexiano

\subsection{Privacy and Security: problems and Solutions}

\bfseries 
Sender authentication:
\normalfont
While the sender (telephone) authentication plays indeed a key role, it’s even
more crucial the authentication of the person behind a control message; that’s
the reason for employing a password based authentication scheme: in the
initialization wizard the user is required to insert a password which is going
to be needed to send a message to that telephone (the basic assumption
is the password is known only by the mobile owner and by some people,
possibly no one, he trusts). The password is stored hashed with SHA-256 in
the application preferences, along with the hashing salt (a random token) to
avoid both time-to-memory attacks (such as rainbow tables) and the equality
of two hashes generated from two equal passwords; the salt is sent to another
telephone after the process of public keys authentication (See section 4.3.6).


\bfseries 
Integrity:
\normalfont
The message received must be exactly the one sent: every transmission error
or tampering must be detected and cause the abort of the current command
session: no retransmission is done.


\bfseries
Authentication:
\normalfont
The receiver must have a secure way to understand which telephone the
received message comes from.


\bfseries 
Non forgeability:
\normalfont
Nobody should be able to forge a command message which is both valid and
correctly authenticated.


\bfseries 
Non Repudation:
\normalfont
The sender must not be able to deny he sent a specific message (if he actually
did it).
Digitally signing every command message can ensure integrity, authentica-
tion, non repudiation and a weak defense against non forgeability: symmetric
encryption (and in particular AES-256 in GCM mode of operation) is needed
for full protection.


\bfseries
Message Confidentiality:
\normalfont
No one should be able to detect that and which command is sent to a mobile
phone, so the command message is encrypted with the symmetric cipher
AES-256 in GCM mode of operation (used for performance reasons and for a
supplementary integrity check).



\bfseries
Asymmetric keys management:
\normalfont
Asymmetric keys management
Digital signatures (and shared secrets computation as we will see) require
asymmetric cryptography: in the initialization wizard the application gener-
ates and stores in the preferences a key pair based on the elliptic curves; the
reasons for this choice are performances and the smaller key length with re-
spect to other keys (like RSA and DSA ones) at a fixed level of security. This
makes the 140 characters (bytes) Android limit for a single sms no more a
problem.


\bfseries
Symmetric key/initialization vector management:
\normalfont
AES-256, being a symmetric cipher, encrypts and decrypts a specific message
with the same key, and, given the communication channel is not secure, the
two parts must agree on the same key in some way; in particular ECDH is
used to compute a common secret once and for all, then, when in need to
send a message, the sender picks up a random 32 bytes salt, forwards it to
the receiver, then both parts use a keyschedule algorithm (PBKDF2 with
HMAC-SHA-256) to derive the same key starting from the secret and the
salt. Furthermore the GCM mode of operation requires for every message the
sender to generate a 12 bytes random initialization vector and to send it to
the receiver.


\bfseries
Public keys mutual authentication:
\normalfont

While dealing with asymmetric cryptography, the main problem is to bind a
public key with a real user to avoid active Man-In-The-Middle (MITM from
now on) attacks. Neither a Pulic Key infrastructure (PKI) or a Web Of Trust
(WOT) is employed, because they are both potentially insecure for various
reasons (in the PKI case the presence of a trusted element, a certification
autorithy hierarchy, which may be compromised/untrusted/fake; in the WOT
case the presence of a net of trusted elements, the ones who signed a specific
public key, which might be fake/bad persons; furthermore a key with no
signatures is not automatically a fake one, but there isn’t a way to tell), so
the application uses a modified version of the Socialist Millionaire Protocol
(SMP) to authenticate to each one each other key; this requires the two parts
to have a common secret (an answer to a particular question set up on the fly
by the users during the SMP), which is easy to achieve, since the two users
are likely to be the same person or two people who trust themselves.

\subsection{Performance}

We chose the crypto algorithms with an eye on the performances of the
whole system: the key idea is the bottleneck must be the sms and not the
computation time required by the encryptions/decryptions; for this reason
the command messages are encrypted with a symmetric algorithm and not
with RSA or ElGamal (or another asymmetric algorithm), since symmetric
cryptography is faster than asymmetric at least by two orders of magnitude
(they are very likely to be 3 anyway); however to do ECDH and ECDSA
the application needs also an asymmetric key pair, which is generate during
the intitial wizard once an for all, so an acceptable overhead. The public
keys mutual validation (SMP + ECDH in practice) takes some time, but it’s
done only one time per association, which means two telephones have to do it
only when they associate themselves. Finally the digital signature/verification
process are quite fast and so is the key-derivation from the secret and the
salt.










