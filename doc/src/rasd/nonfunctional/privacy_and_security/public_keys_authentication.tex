\paragraph{Public keys mutual authentication} \hspace{0pt} \\
\small{While dealing with asymmetric cryptography, the main problem is to bind a public key with a real user to avoid active Man-In-The-Middle (MITM from now on) attacks. Neither a Pulic Key infrastructure (PKI) or a Web Of Trust (WOT) is employed, because they are both potentially insecure for various reasons (in the PKI case the presence of a trusted element, a certification autorithy hierarchy, which may be compromised/untrusted/fake; in the WOT case the presence of a net of trusted elements, the ones who signed a specific public key, which might be fake/bad persons; furthermore a key with no signatures is not automatically a fake one, but there isn't a way to tell), so the application uses a modified version of the Socialist Millionaire Protocol (SMP) to authenticate to each one each other key; this requires the two parts to have a common secret (an answer to a particular question set up on the fly by the users during the SMP), which is easy to achieve, since the two users are likely to be the same person or two people who trust themselves.}